Our research contributes in several ways. First and foremost, we provide a novel method for calculating the intraday CPIV with moneyness and maturity adjusted weights term. Also, we aggregate the sample data across a business cycle including the 2008 financial crisis to validate the stability of our empirical results. Next, we examine the contemporaneous and intertemporal relations between CPIV and expected market returns. In our findings, the open intervals and mid intervals of quote-data CPIV are significantly positively related to the contemporaneous index returns. However, we do not find strong evidence that the intertemporal relation exists. Therefore, if further researchers want to investigate the intertemporal relations between the options market and index market, we suggest to employ volatility skewness rather than implied volatility spreads.   

Besides, we also run a battery of tests in intraday SPX index returns on distinct interval CPIV. In empirical results, the open 5-minutes interval CPIV has strong predictability on intraday half-hours cumulated returns. Moreover, the predictive power decreases as hours decay during the remainder of the trading day. We claim that the most informational interval is the first 5-minute interval no matter on the daily frequency and intraday frequency. 

Last but not least, we provide several option explanations for the relationship between CPIV and expected index returns. During the periods with high consumer sentiment, the documented relationship is significantly more pronounced compared with the rest of the periods. We also find that given the effect of option liquidity, the relationship becomes formidable during high option liquidity periods, and the results are consistent with the effect of economic states.  