

\subsection{Relationships Between CPIV and Option Characteristic}
Before we explore the relationship between the index returns and CPIV, we need to survey on the option characteristics. From equation \ref{eq:char}, We expect the coefficient of time non-synchronization, moneyness. and maturity to be positive. There should be no deviation in put-call parity given that the underlying price is also identical in theroy. In other words, the larger the time gap between the pair option, the larger the divergence may occur and cause higher CPIV in absolute form. Furthermore, the level of moneyness and time to expiration are crucial to CPIV as well. According to \textcite{hentschel2003errors}, implied volatilities from options away from the money are especially sensitive to measurement errors in option prices and uderlying asset price. Therefore, we estimates that with greater level of moneyness and time to maturities, the measurement errors in implied volatilities would increase. 

\subsection{Relationships Between Index Return and CPIV}