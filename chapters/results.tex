
The main results are based on SPX option quote data. We take both \autoref{eq: withoutadj} (non-adjusted weights) and \autoref{eq: withadj} (adjusted weights) to build intraday CPIV. The results are quite similar, so we only present the results of \autoref{eq: withadj}. The non-adjusted version is upon requsted.

\subsection{Relationships Between CPIV and Option Characteristic}
Before we explore the relationship between the index returns and CPIV, we need to survey on the option characteristics. From \autoref{eq:char}, We expect the coefficient of time non-synchronization, moneyness. and maturity to be positive. There should be no deviation in put-call parity given that the underlying price is also identical in theroy. In other words, the larger the time gap between the pair option, the larger the divergence may occur and cause higher CPIV in absolute form. Furthermore, the level of moneyness and time to expiration are crucial to CPIV as well. According to \textcite{hentschel2003errors}, implied volatilities from options away from the money are especially sensitive to measurement errors in option prices and uderlying asset price. Therefore, we estimates that with greater level of moneyness and time to maturities, the measurement errors in implied volatilities would increase. 

The results in \autoref{table:regression1} confirm our earlier statements based on regreesion of CPIV. Column (1) reports the regression of CPIV on time non-synchronization. All the coefficients are highly significant; they imply a positive effect on deviation of put-call parity. Column (2) of table 5 shows that regression of CPIV on moneyness. The coefficient of moneyness term is also highly siginficant (0.3833, t-statistic 434.96). Column (3) signifies of CPIV on maturity. The coefficient of maturity term is highly siginficant (0.0348, t-statistic 296.27) as well. Clearly, the regression results are aligned with previous literatures, which indicate the moneyness and maturity have positive effect on CPIV. Finally, in column (4), we regress CPIV on all option pair characteristics. The outcome is similar to previous regressions. Noticeably, the adjusted $R^{2}$ of moneyness term is almost 34\%, which stands for the most influential factor. 

\subsection{Relationships Between Index Return and CPIV}
In this subsection, we discuss the relationship between CPIV and index returns in different aspects. The Panel A in \autoref{table:regression2} depict the results of contemporaneous index return regressions, while Panel B acts for the results of one-day ahead index return regressions. The header of columns describes the interval of CPIV, and all th variables are in percentage format. 

In panel A, the CPIV coefficients are all positive and this outcome is aligned with \textcite{cremers2010deviations} which proves the evidence for a significant positive link between implied volatility spread and expected returns. To be more specific, $CPIV_{08:30}$ and $CPIV_{12:00}$ are significant in 1\% and 10\% level respectively. However, other intervals are not significant enough under 10\% level. The result validate our first hypothesis: The open and mid intervals may also contain important information toward index returns. Apart from that, $CPIV_{08:30}$ has the largest coeffecient 8.33\footnote{The value is shown in percentage format.} (t-statistic 17.84), we believe that most trading information is incoporated into the first interval.

In panel B, we test the next day index return on CPIV and other control variables to see whether the information spillover from the option market to the next-day index market. Surprisingly, none of them are significant enough and not all of them are positive. According to \textcite{atilgan2015implied}, they assert the volatilty skewness of S\&P index option may cause spillover effect to index market. We find out the implied volatility spread has no predictability on aggregate index returns in one-day horizon. This is slightly different from \textcite{atilgan2015implied}, since we choose implied volatilty spreads rather than volatilty skewness as the measures. The implied volatility spreads absorb all possible option pairs information while volatilty skewness only include the information from out-of-money (OTM) put options and at-the-money (ATM) call options. Briefly, the implied volatilty spreads may be too noisy in forcasting future index returns.  

On intra-day frequency, we also test the predictability on culumlated half hour index returns in different horizons. We only take $CPIV_{08:30}$ and $CPIV_{12:00}$ as independent variables since they are the only variables significant on contemporaneous frequency. However, the results are completely not the same way. From \autoref{table:regression4}, the coefficients of $CPIV_{08:30}$ range from 5.63 to 6.57, implying myriad economic significance. The adjusted $R^{2}$ of $CPIV_{08:30}$ range from 50.19\% to 19.17\% as hours decay. The results is consistent with our Hypothesis 2: The first 5-minutes CPIV has predictability on intra-day index return during the rest of the trading day. On the contrary, we did not find prediction power on $CPIV_{12:30}$ for any k. In this paragraph, we demonstrate that the first 5-minutes interval keep considerable information compared to other intervals, and this information would be incorporated into index market by hours -- not by days. Our emperical results are sligtly different from \textcite{cremers2010deviations}, but similar to \textcite{kumar1992behavior}. 
   




%====================================================================
%                        Tables 5
%====================================================================

\begin{table}[h]
\centering
\caption{Regression Results: CPIV and Option Pair Characterisitcs}\label{table:regression1}
\begin{threeparttable}

\medskip

{\scriptsize 
This table shows the regression results of call-put implied volatility (CPIV) on moneyness, time non-synchronization, maturity with intervals fixed effect.The CPIV term and moneyness term are in absolute form due to the opposite sign of deviation may eliminate the effect of each other. TimeDiff represents time non-synchronization. Moneyness stands for the level of divergence between the underlying price and the excersie price. Maturity symbolizes the maturity date. As
for Interval F.E., we control the intervals effect. All the t statistics in parathesis are Newey-West t statistics adjusted.
}
\medskip
\small
\centering
\begin{tabular}{ccccc}
\toprule
              & (1)        & (2)        & (3)        & (4)        \\ \midrule
Intercept     & 0.05577*** & 0.03087*** & 0.05055*** & 0.03011*** \\
              & (263.91)   & (170.91)   & (229.42)   & (162.73)   \\
TimeDiff      & 0.0000***  &            &            & 0.0000***  \\
              & (89.32)    &            &            & (7.40)     \\
Moneyness     &            & 0.3833***  &            & 0.3710***  \\
              &            & (434.96)   &            & (366.56)   \\
Maturity      &            &            & 0.0348***  & 0.0103***  \\
              &            &            & (296.27)   & (68.17)    \\
Interval F.E. & Yes        & Yes        & Yes        & Yes        \\
No. Obs       & 1692542    & 1692542    & 1692542    & 1692542    \\
Adj. $R^{2}$  & 0.0473     & 0.3368     & 0.0853     & 0.3402     \\
\bottomrule
\end{tabular}

\begin{tablenotes}
%\multicolumn{4}{l}{\footnotesize \textit{Newey-West t} statistics in parentheses} \\
%\multicolumn{4}{l}{\footnotesize * $p < 0.10$, ** $p < 0.05$, *** $p < 0.01$} \\
\item
\item[***]Significant at the 1 percent level.    
\item[**]Significant at the 5 percent level.   
\item[*]Significant at the 10 percent level.
\end{tablenotes}

\end{threeparttable}

\end{table}


%====================================================================
%                        Tables 6, 7
%====================================================================



\begin{table}[h]

\caption{Regression Results: Index Return Predictability on CPIV of Quote Data}\label{table:regression2}
\begin{threeparttable}

\medskip

{\scriptsize 
This table reports from time-series predictive regressions of daily return of the S\&P 500 index on call-put implied volatility (CPIV), and macroeconomic varibles. Panel A presents the summary statistics for contemporaneous index return regression. Panel B presents summary statistics for one-day ahead index return regression. CPIV elucidates the interval implied volatility spreads.DEF explicates the change in the difference between the yeilds on BAA- and AAA-rated coporate bonds. TERM expounds the difference between the yeilds on the 10-year Treasury bond and one-month Treasury bill. All the t statistics in parathesis are Newey-West t statistics adjusted. All variable values are in percentage format.  
}
\medskip

\begin{subtable}[t]{\linewidth}

\caption{Panel A: The Contemporaneous Index Return Regression }
\tiny
\begin{tabular}{ccccccccccccccc}
\toprule
	& 08:30   & 09:00   & 09:30   & 10:00   & 10:30   & 11:00   & 11:30   & 12:00   & 12:30   & 13:00   & 13:30   & 14:00   & 14:30   & 15:00   \\ \midrule
Intercept & 0.26*** & 0.15    & 0.17    & 0.16    & 0.14    & 0.21*   & 0.13    & 0.25**  & 0.17    & 0.18    & 0.14    & 0.16    & 0.27**  & 0.31*   \\
      & (2.84)  & (1.47)  & (1.62)  & (1.52)  & (1.32)  & (1.82)  & (1.18)  & (2.07)  & (1.36)  & (1.38)  & (1.02)  & (1.11)  & (1.90)  & (2.04)  \\
CPIV      & 8.33*** & 1.35    & 1.30    & 1.84    & 1.50    & 2.86    & 0.41    & 3.86*   & 2.24    & 2.13    & 1.03    & 2.57    & 4.91    & -0.25   \\
      & (17.84) & (0.85)  & (0.83)  & (1.07)  & (0.85)  & (1.48)  & (0.20)  & (1.83)  & (1.01)  & (1.09)  & (0.51)  & (1.21)  & (1.91)  & (-0.09) \\
DEF       & 0.04    & -0.06   & -0.07   & -0.06   & -0.06   & -0.09   & -0.10   & -0.07   & -0.07   & -0.09   & -0.11   & -0.09   & -0.12   & -0.28   \\
      & (0.52)  & (-0.63) & (-0.70) & (-0.61) & (-0.60) & (-0.73) & (-0.81) & (-0.55) & (-0.54) & (-0.73) & (-0.83) & (-0.67) & (-0.83) & (-1.77) \\
TERM      & -0.03   & 0.00    & 0.00    & 0.02    & 0.02    & 0.02    & 0.02    & 0.01    & 0.02    & 0.03    & 0.04    & 0.04*   & 0.03    & 0.00    \\
      & (-1.29) & (0.07)  & (0.21)  & (0.71)  & (0.92)  & (0.77)  & (1.00)  & (0.59)  & (0.98)  & (1.26)  & (1.63)  & (1.74)  & (1.33)  & (0.00)  \\
      &         &         &         &         &         &         &         &         &         &         &         &         &         &         \\
Adj. $R^{2}$    & 24.33   & 0.12    & 0.15    & 0.17    & 0.16    & 0.33    & 0.20    & 0.43    & 0.24    & 0.32    & 0.30    & 0.43    & 0.84    & 1.27  \\
\bottomrule
\end{tabular}

\begin{tablenotes}
%\multicolumn{4}{l}{\footnotesize \textit{Newey-West t} statistics in parentheses} \\
%\multicolumn{4}{l}{\footnotesize * $p < 0.10$, ** $p < 0.05$, *** $p < 0.01$} \\
\item
\item[***]Significant at the 1 percent level.    
\item[**]Significant at the 5 percent level.   
\item[*]Significant at the 10 percent level.
\end{tablenotes}
\end{subtable}

\medskip
\begin{subtable}[t]{\linewidth}

\caption{Panel B: The One-Day ahead Index Return Regression }
\tiny
\begin{tabular}{ccccccccccccccc}
\toprule
          & 08:30   & 09:00   & 09:30   & 10:00   & 10:30   & 11:00  & 11:30   & 12:00  & 12:30   & 13:00   & 13:30   & 14:00   & 14:30   & 15:00   \\ \midrule
Intercept & 0.02    & 0.04    & 0.12    & 0.03    & 0.06    & 0.02   & 0.02    & 0.07   & 0.08    & 0.13    & 0.08    & 0.11    & 0.15    & 0.13    \\
          & (0.20)  & (0.44)  & (1.18)  & (0.31)  & (0.56)  & (0.15) & (0.18)  & (0.61) & (0.66)  & (1.08)  & (0.66)  & (0.84)  & (1.10)  & (0.91)  \\
CPIV      & -0.74   & -0.54   & 1.24    & -1.65   & 0.15    & 0.75   & -1.11   & 1.31   & -0.10   & 0.81    & -0.64   & -0.18   & -0.49   & -2.00   \\
          & (-1.23) & (-0.36) & (0.80)  & (-0.87) & (0.08)  & (0.40) & (-0.57) & (0.70) & (-0.05) & (0.42)  & (-0.31) & (-0.09) & (-0.22) & (-0.84) \\
DEF       & -0.01   & -0.04   & -0.05   & -0.07   & -0.04   & 0.03   & -0.05   & 0.00   & -0.07   & -0.09   & -0.09   & -0.10   & -0.14   & -0.15   \\
          & (-0.15) & (-0.39) & (-0.53) & (-0.67) & (-0.35) & (0.29) & (-0.48) & (0.01) & (-0.61) & (-0.75) & (-0.77) & (-0.83) & (-1.13) & (-1.07) \\
TERM      & 0.00    & 0.00    & 0.00    & 0.01    & 0.01    & 0.01   & 0.02    & 0.01   & 0.01    & 0.01    & 0.01    & 0.01    & 0.00    & -0.01   \\
          & (0.14)  & (0.11)  & (-0.01) & (0.29)  & (0.33)  & (0.37) & (0.75)  & (0.39) & (0.42)  & (0.41)  & (0.36)  & (0.28)  & (0.09)  & (-0.19) \\
          &         &         &         &         &         &        &         &        &         &         &         &         &         &         \\
Adj. $R^{2}$     & 0.19    & 0.02    & 0.10    & 0.08    & 0.02    & 0.03   & 0.05    & 0.03   & 0.07    & 0.13    & 0.09    & 0.12    & 0.24    & 0.31 \\ 
\bottomrule
\end{tabular}

\begin{tablenotes}
%\multicolumn{4}{l}{\footnotesize \textit{Newey-West t} statistics in parentheses} \\
%\multicolumn{4}{l}{\footnotesize * $p < 0.10$, ** $p < 0.05$, *** $p < 0.01$} \\
\item
\item[***]Significant at the 1 percent level.    
\item[**]Significant at the 5 percent level.   
\item[*]Significant at the 10 percent level.
\end{tablenotes}
\end{subtable}


\end{threeparttable}

\end{table}

%====================================================================
%                        Tables 8
%====================================================================



\begin{table}[h]

\caption{Regression Results: Intra-day Index Return Predictability on CPIV of Quote Data}\label{table:regression4}
\begin{threeparttable}

\medskip
\begin{spacing}{1.1}
{\scriptsize  
This table records from time-series predictive regressions of intra-day half-hour return of the S\&P 500 index on call-put implied volatility (CPIV), and macroeconomic varibles. The independent varible is $CPIV_{08:30}$. DEF explicates the change in the difference between the yeilds on BAA- and AAA-rated coporate bonds. TERM expounds the difference between the yeilds on the 10-year Treasury bond and one-month Treasury bill. $K$ means k of half-hour ahead cumulative returns. For example, k = 1 means the cumulative returns from 08:30 to 09:00, k = 2 means the cumulative returns from 08:30 to 09:30, and so on and so forth. All the t statistics in parathesis are Newey-West t statistics adjusted. All variable values are in percentage format. 
}
\end{spacing}
\medskip
\tiny

\begin{tabular}{ccccccccccccccc}
\toprule

          & K=1     & K=2     & K=3     & K=4     & K=5     & K=6     & K=7     & K=8     & K=9     & K=10    & K=11    & K=12    & K=13    \\ \midrule
Intercept & 0.18*** & 0.16*** & 0.16*** & 0.19*** & 0.19*** & 0.20*** & 0.20*** & 0.21*** & 0.22*** & 0.22*** & 0.17*** & 0.20*** & 0.18*** \\
          & (5.77)  & (4.11)  & (3.52)  & (3.79)  & (3.58)  & (3.74)  & (3.53)  & (3.44)  & (3.40)  & (3.24)  & (2.27)  & (2.61)  & (2.16)  \\
$CPIV_{08:30}$& 5.63*** & 5.77*** & 5.79*** & 5.80*** & 5.75*** & 5.81*** & 5.83*** & 5.93*** & 5.87*** & 5.94*** & 5.96*** & 6.12*** & 6.57*** \\
          & (30.81) & (27.76) & (24.92) & (24.37) & (21.44) & (22.27) & (21.92) & (22.05) & (20.71) & (19.94) & (19.12) & (17.58) & (16.17) \\
DEF       & 0.00    & 0.02    & 0.03    & 0.00    & 0.01    & 0.00    & 0.00    & -0.01   & -0.03   & -0.04   & 0.01    & 0.01    & 0.03    \\
          & (-0.13) & (0.67)  & (0.64)  & (0.06)  & (0.23)  & (-0.04) & (-0.05) & (-0.26) & (-0.60) & (-0.62) & (0.22)  & (0.09)  & (0.45)  \\
TERM      & -0.01   & -0.01   & -0.02   & -0.02   & -0.02   & -0.02   & -0.02   & -0.02   & -0.01   & -0.01   & -0.01   & -0.01   & -0.02   \\
          & (-1.58) & (-1.43) & (-1.39) & (-1.54) & (-1.81) & (-1.57) & (-1.45) & (-1.07) & (-0.73) & (-0.70) & (-0.43) & (-0.81) & (-0.83) \\
          &         &         &         &         &         &         &         &         &         &         &         &         &         \\
Adj.$R^{2}$    & 50.19   & 41.65   & 36.43   & 33.14   & 29.80   & 28.79   & 27.25   & 26.43   & 24.69   & 23.11   & 20.85   & 19.87   & 19.37   \\

\bottomrule
\end{tabular}

\begin{tablenotes}
%\multicolumn{4}{l}{\footnotesize \textit{Newey-West t} statistics in parentheses} \\
%\multicolumn{4}{l}{\footnotesize * $p < 0.10$, ** $p < 0.05$, *** $p < 0.01$} \\
\item
\item[***]Significant at the 1 percent level.    
\item[**]Significant at the 5 percent level.   
\item[*]Significant at the 10 percent level.
\end{tablenotes}


\end{threeparttable}

\end{table}


%====================================================================
%                        Tables 9, 10
%====================================================================


%\begin{table}[h]
%
%\caption{Regression Results: Index Return Predictability on CPIV of Trade Data}\label{trade_table:regression2}
%\begin{threeparttable}
%
%\medskip
%
%{\scriptsize 
%This table reports from time-series predictive regressions of daily return of the S\&P 500 index on call-put implied volatility (CPIV), and macroeconomic varibles. Panel A presents the summary statistics for contemporaneous index return regression. Panel B presents summary statistics for one-day ahead index return regression. CPIV elucidates the interval implied volatility spreads.DEF explicates the change in the difference between the yeilds on BAA- and AAA-rated coporate bonds. TERM expounds the difference between the yeilds on the 10-year Treasury bond and one-month Treasury bill. All the t statistics in parathesis are Newey-West t statistics adjusted. All variable values are in percentage format.  
%}
%\medskip
%
%\begin{subtable}[t]{\linewidth}
%
%\caption{Panel A: The Contemporaneous Index Return Regression }
%\tiny
%\begin{tabular}{ccccccccccccccc}
%\toprule
%             & 08:30   & 09:00   & 09:30   & 10:00   & 10:30   & 11:00   & 11:30   & 12:00   & 12:30   & 13:00   & 13:30   & 14:00   & 14:30   & 15:00   \\ \midrule
%Intercept       & 0.22    & 0.01    & -0.04   & 0.03    & -0.03   & 0.03    & 0.01    & 0.01    & 0.00    & 0.05    & 0.04    & 0.01    & -0.05   & 0.01    \\
%             & (2.48)  & (0.06)  & (-0.39) & (0.35)  & (-0.31) & (0.30)  & (0.12)  & (0.05)  & (-0.04) & (0.43)  & (0.37)  & (0.10)  & (-0.42) & (0.10)  \\
%CPIV            & 16.01   & -3.41   & -6.36   & -2.28   & -5.98   & -3.61   & -6.48   & -4.19   & -2.39   & 0.13    & -0.13   & -1.07   & -6.13   & -3.07   \\
%             & (15.91) & (-0.98) & (-2.21) & (-0.98) & (-4.17) & (-1.46) & (-2.00) & (-1.53) & (-1.30) & (0.04)  & (-0.04) & (-0.33) & (-1.95) & (-1.36) \\
%DEF             & 0.05    & -0.08   & -0.09   & -0.09   & -0.09   & -0.11   & -0.13   & -0.07   & -0.07   & -0.06   & -0.08   & -0.06   & -0.10   & -0.09   \\
%             & (0.63)  & (-0.82) & (-0.97) & (-0.98) & (-1.01) & (-1.15) & (-1.32) & (-0.82) & (-0.67) & (-0.65) & (-0.87) & (-0.60) & (-1.04) & (-0.97) \\
%TERM            & 0.01    & 0.02    & 0.02    & 0.02    & 0.02    & 0.02    & 0.01    & 0.01    & 0.02    & 0.02    & 0.03    & 0.02    & 0.02    & 0.03    \\
%             & (0.50)  & (0.89)  & (0.95)  & (1.04)  & (0.87)  & (0.85)  & (0.48)  & (0.47)  & (0.95)  & (0.96)  & (1.21)  & (0.98)  & (1.09)  & (1.25)  \\
%Adj. $R^{2}$ & 26.39   & 0.32    & 0.27    & 0.26    & 0.26    & 0.53    & 0.25    & 0.60    & 0.30    & 0.63    & 0.53    & 0.68    & 1.38    & 1.69     \\
%\bottomrule
%\end{tabular}
%
%\begin{tablenotes}
%%\multicolumn{4}{l}{\footnotesize \textit{Newey-West t} statistics in parentheses} \\
%%\multicolumn{4}{l}{\footnotesize * $p < 0.10$, ** $p < 0.05$, *** $p < 0.01$} \\
%\item
%\item[***]Significant at the 1 percent level.    
%\item[**]Significant at the 5 percent level.   
%\item[*]Significant at the 10 percent level.
%\end{tablenotes}
%\end{subtable}
%
%\medskip
%\begin{subtable}[t]{\linewidth}
%
%\caption{Panel B: The One-Day ahead Index Return Regression }
%\tiny
%\begin{tabular}{ccccccccccccccc}
%\toprule
%             & 08:30   & 09:00   & 09:30   & 10:00   & 10:30   & 11:00   & 11:30   & 12:00   & 12:30   & 13:00  & 13:30   & 14:00   & 14:30   & 15:00   \\ \midrule
%Intercept       & 0.00    & 0.00    & 0.02    & 0.06    & 0.16    & 0.07    & 0.00    & 0.01    & 0.08    & 0.00   & 0.06    & 0.01    & 0.04    & 0.03    \\
%             & (-0.02) & (0.04)  & (0.19)  & (0.66)  & (1.46)  & (0.68)  & (-0.03) & (0.09)  & (0.74)  & (0.04) & (0.61)  & (0.06)  & (0.37)  & (0.25)  \\
%CPIV            & -2.19   & -2.85   & -0.64   & 0.70    & 4.65*** & 0.08    & -3.65   & -4.17   & 0.63    & 0.89   & 4.96    & -1.94   & -0.84   & -0.62   \\
%             & (-1.69) & (-0.91) & (-0.20) & (0.23)  & (2.00)  & (0.03)  & (-1.24) & (-1.26) & (0.61)  & (0.24) & (1.38)  & (-0.97) & (-0.30) & (-0.67) \\
%DEF             & -0.02   & -0.03   & -0.01   & -0.02   & -0.03   & -0.04   & -0.03   & -0.05   & -0.06   & 0.04   & 0.06    & -0.01   & -0.02   & 0.00    \\
%             & (-0.18) & (-0.30) & (-0.07) & (-0.23) & (-0.35) & (-0.41) & (-0.34) & (-0.53) & (-0.59) & (0.44) & (0.68)  & (-0.16) & (-0.21) & (0.04)  \\
%TERM            & 0.00    & 0.00    & 0.00    & 0.00    & -0.01   & 0.00    & 0.00    & 0.00    & 0.01    & 0.00   & 0.00    & 0.01    & 0.00    & 0.00    \\
%             & (0.21)  & (0.06)  & (0.09)  & (-0.07) & (-0.28) & (-0.17) & (-0.15) & (-0.20) & (0.24)  & (0.07) & (-0.10) & (0.28)  & (-0.13) & (-0.20) \\
%             &         &         &         &         &         &         &         &         &         &        &         &         &         &         \\
%Adj. $R^{2}$ & 0.52    & 0.10    & 0.00    & 0.02    & 0.71    & 0.03    & 0.17    & 0.21    & 0.07    & 0.04   & 0.41    & 0.09    & 0.01    & 0.03   \\ 
%\bottomrule
%\end{tabular}
%
%\begin{tablenotes}
%%\multicolumn{4}{l}{\footnotesize \textit{Newey-West t} statistics in parentheses} \\
%%\multicolumn{4}{l}{\footnotesize * $p < 0.10$, ** $p < 0.05$, *** $p < 0.01$} \\
%\item
%\item[***]Significant at the 1 percent level.    
%\item[**]Significant at the 5 percent level.   
%\item[*]Significant at the 10 percent level.
%\end{tablenotes}
%\end{subtable}
%
%
%\end{threeparttable}
%
%\end{table}
%

