
The main results are based on SPX option quote data. We take both \autoref{eq: withoutadj} (non-adjusted weights) and \autoref{eq: withadj} (adjusted weights) to build intraday CPIV. The results are quite similar, so we only present the results of \autoref{eq: withadj}. The non-adjusted version is upon request.

\subsection{Relationships Between CPIV and Option Characteristic}
Before we explore the relationship between the index returns and CPIV, we need to survey on the option characteristics. From \autoref{eq:char}, We expect the coefficient of time non-synchronization, moneyness. and maturity to be positive. There should be no deviation in put-call parity given that the underlying price is identical in theory. In other words, the larger the time gap between the pair option, the larger the divergence may occur and cause higher CPIV in absolute form. Furthermore, the level of moneyness and time to expiration are crucial to CPIV as well. According to \textcite{hentschel2003errors}, implied volatilities from options away from the money are especially sensitive to measurement errors in option prices and underlying asset price. Therefore, we estimate that the greater level of moneyness and time to maturities, the measurement errors in implied volatilities to increase. 

The results in \autoref{table:regression1} confirm our earlier statements based on regression of CPIV. Column (1) reports the regression of CPIV on time non-synchronization. All the coefficients are highly significant; they imply a positive effect on the deviation of put-call parity. Column (2) of table 5 shows that regression of CPIV on moneyness. The coefficient of moneyness term is also highly significant (0.3833, t-statistic 434.96). Column (3) signifies of CPIV on maturity. The coefficient of maturity term is highly significant (0.0348, t-statistic 296.27) as well. Clearly, the regression results are aligned with previous literature, which indicates that the moneyness and maturity have a positive effect on CPIV. Finally, in column (4), we regress CPIV on all option pair characteristics. The outcome is similar to previous regressions. Noticeably, the adjusted $R^{2}$ of moneyness term is almost 34\%, which stands for the most influential factor.  

\subsection{Relationships Between Index Return and CPIV}
In this subsection, we discuss the relationship between CPIV and index returns in different aspects. Panel A in \autoref{table:regression2} depict the results of contemporaneous index return regressions, while Panel B acts as the results of one-day ahead index return regressions. The header of columns describes the interval of CPIV, and all the variables are in a percentage format. 

In panel A, the CPIV coefficients are all positive and this outcome is aligned with \textcite{cremers2010deviations} which proves the evidence for a significant positive link between implied volatility spread and expected returns. To be more specific, $CPIV_{08:30}$ and $CPIV_{12:00}$ are significant in 1\% and 10\% level respectively. However, other intervals are not significant enough under 10\% level. The result validates our first hypothesis: The open and mid intervals may also contain important information toward index returns. Apart from that, $CPIV_{08:30}$ has the largest coefficient 8.33\footnote{The value is shown in a percentage format.} (t-statistic 17.84), we believe that most trading information and news of previous days get incorporated into the first interval.

In panel B, we test the one day ahead index returns on CPIV and other control variables to see whether the information spillover from the options market to one day ahead index market. Surprisingly, none of them is significant enough and not all of them are positive. According to \textcite{atilgan2015implied}, they assert that the volatility skewness of S\&P index option may cause a spillover effect to the index market. We find that the implied volatility spreads have no predictability on aggregate index returns in the one-day horizon. This is slightly different from \textcite{atilgan2015implied} since we choose implied volatility spreads rather than volatility skewness as the measures. The implied volatility spreads absorb all possible option pairs information while volatility skewness only include the information from out-of-money (OTM) put options and at-the-money (ATM) call options. Briefly, the implied volatility spreads may be too noisy in forecasting future index returns.  

On intraday frequency, we also test the predictability on cumulative half hour index returns in different horizons. We only take $CPIV_{08:30}$ and $CPIV_{12:00}$ as independent variables since they are the only variables significant on daily frequency. However, the results are completely not the same way. From \autoref{table:regression4}, the coefficients of $CPIV_{08:30}$ range from 5.63 to 6.57, implying myriad economic significance. The adjusted $R^{2}$ of $CPIV_{08:30}$ range from 50.19\% to 19.17\% as hours decay. The results are consistent with our Hypothesis 2: The first 5-minutes CPIV has predictability on intraday index cumulated returns for the remainder of the same trading day. On the contrary, we did not find predictive power on $CPIV_{12:30}$ for any $k$. In this paragraph, we demonstrate that the first 5-minutes interval keeps considerable information compared to other intervals, and this information would be incorporated into the index market by hours -- not by days. Our empirical results are slightly different from \textcite{cremers2010deviations}, but similar to \textcite{kumar1992behavior}. 
   





