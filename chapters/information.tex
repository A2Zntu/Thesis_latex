
\subsection{Consumer Sentiment}
In this section, we provide evidence for the intertemporal relationship between CPIV from the options market and index future returns. 
According to \textcite{atilgan2015implied}, the periods of extremely high or low consumer sentiment index are crucial because these are the periods where asset prices could deviate from the fundamental values intensely. Therefore, one would expect, during periods of extreme consumer sentiment, the intertemporal relation between CPIV and index returns to be stronger.
 \begin{equation}  \label{eq:info}
SPX\_Return_{t} = \alpha + \beta _{1}VSPLUS_{t} + \beta _{2}VSMINUS_{t} + \beta _{3}DEF_{t} + \beta _{4}TERM_{t} + \varepsilon _{t}
 \end{equation}
We follow the previous method to define a dummy variable that is equal to one for a given trading day if the consumer sentiment index is greater than its $90^{th}$ percentile or less than its $10^{th}$ percentile over the sample period, and 0 otherwise. Hence, following prior studies, VSPLUS is denoted as CPIV if dummy variable is equal to one and 0 otherwise, while VSMINUS is denoted as CPIV if dummy variable is equal to zero and 0 otherwise. If the information in CPIV to consumer sentiment is essential in the predictive power of the CPIV on index returns, then $\beta _{1}$ is expected to be positive and larger than $\beta _{2}$. 

\autoref{table:BW_sentiment} focuses on Baker and Wurgler (BW) sentiment index\footnote{The data can be found in Jeffrey Wurgler personal website}, while \autoref{table:UOM_sentiment} centers around University of Michigan Consumer (UOM) Sentiment Index\footnote{The data are available on the University of Michigan Website}. Firstly, from \autoref{table:BW_sentiment}, our empirical results show that all the coefficients of VSPLUS are positive and greater than the coefficients of VSMINUS except for the open-interval. Moreover, compared with the original contemporaneous \autoref{eq:contem}, there are more than two intervals with significant positive coefficients, and 1.2 basis higher than original coefficients on average. 

Secondly, from \autoref{table:UOM_sentiment}, the results based on UOM sentiment measures are consistent with those based on the BW sentiment measure. Not only all the coefficients of VSPLUS are positive and greater than the coefficients of VSMINUS, but also there are even more significant VSPLUS than the amount of BW sentiment measure. From the table, the coefficients of VSPLUS of the significant interval is around 0.05\%, which are 1.5 times larger than the coefficients of the original regressions. 

In summary, these results are consistent with \textcite{baker2006investor}, which argues that prices can deviate from fundamental values because a significant part of the investor class is irrational at the roots of behavioral finance. Hence, the predictability produced by CPIV is found to be more noticeable during the high sentiment periods.  

\subsection{Option Liquidity}

Myriad studies like \textcite{easley1998option}, \textcite{cremers2010deviations}, \textcite{driessen2012option} indicated that the predictability will be stronger in options with higher liquidity. In this section, we also follow the method in \textcite{chang2018implied} to segregate the high option liquidity periods and others. We use the S\&P 500 option daily trading volume to represent the option liquidity, which is available on WRDS. Furthermore, we define a dummy variable that is equal to one for a given trading day if the options volume is greater than its $90^{th}$ percentile over the sample period, and 0 otherwise. Hence, follow prior studies, VSPLUS is denoted as CPIV if dummy variable is equal to one and 0 otherwise, while VSMINUS is denoted as CPIV of dummy variable is equal to zero and 0 otherwise. Finally, if the information in CPIV to option liquidity is instrumental in the predictive power of the CPIV on index returns, then $\beta _{1}$ is expected to be positive and larger than $\beta _{2}$.

\autoref{table:Volume} describes the regression results on the effect of option liquidity. All the coefficients of VSPLUS are positive and greater than the coefficients of VSMINUS. Moreover, compared with the original contemporaneous \autoref{eq:contem}, there are five intervals with significant positive coefficients, and 3.4 basis higher than original coefficients on average. 

In summary, we find that the predictability of contemporaneous index return increases with option liquidity. These results are in line with previous literature. 

\subsection{Economic States}

We also test regressions on the effect of economic states. The economic states are segregated according to NBER business cycles. We believe that during recession time, the stock market liquidity is relatively lower than the options market liquidity since there are short-sell constraints in the stock market. Moreover, our underlying -- SPX is not available to short while the economy is going down. In contrast, investors could buy a put option to hedge the risk during the recession. Therefore, we believe that the results of regressions given segregated economic states would be consistent with the sentiment measures and option liquidity.  

We utilize economic states from NBER to set apart the overall sample. Then, we define a dummy variable that is equal to one for a given trading day if the economic state bumps into a recession, and 0 if it comes into expansion. Hence, follow prior studies, VSPLUS is denoted as CPIV if dummy variable is equal to one and 0 otherwise, while VSMINUS is denoted as CPIV of dummy variable is equal to zero and 0 otherwise. If the information in CPIV to economic is critical in the predictive power of the CPIV on index returns, then $\beta _{1}$ is expected to be positive and larger than $\beta _{2}$.

\autoref{table:EcoStates} describes the regression results on the effect of economic states. All the coefficients of VSPLUS are positive and greater than the coefficients of VSMINUS. Moreover, compared with the original contemporaneous \autoref{eq:contem}, there are seven intervals with significant positive coefficients, including open intervals, mid intervals, and close intervals. 

In a nutshell, we suggest that during the recession, prices and volume from the option market are much more informational toward the index market since the periods of consumer sentiments and the periods of option liquidity have high correlations with the periods of economic states.  












