\section*{\centerline{\textbf{\huge Abstract}}}
\addcontentsline{toc}{section}{Abstract} 
\noindent


Numerous studies have investigated the relationship between call-put implied volatility spreads (CPIV) and expected equity returns on daily frequency. While we provide a novel method for calculating the intraday CPIV with moneyness and maturity adjusted weight in order to examine the explanatory power. We observe that the CPIV of open market intervals and mid intervals have both positive links to S\&P 500 contemporaneous index returns. This relationship is significantly more intense for the periods during which (i) consumer sentiment measures reach extreme values (ii) option liquidity is relatively high (iii) economic states head to a downturn. In addition, we also examine the predictability of intraday CPIV on intraday half-hours cumulated returns, discovering the open market interval may contain the most information in a single trading day and the predictability decreases as hours decay during the remainder of a day.
\vspace{1cm}


\noindent{\bf Keywords:} S\&P 500; Option returns; Implied volatility spreads; Intraday data; Investors sentiment; Option liquidity