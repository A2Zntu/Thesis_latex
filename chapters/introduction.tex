
In recent year, the information of derivatives market plays an increasingly crucial role in financial markets. There are multiple reasons why option market are vital for investors. One, informed traders may choose to trade in derivative markets since they can camouflage with noise traders by trading different option contracts on one specific security \parencite{easley1998option}. Another, authors such as \textcite{black1973pricing}, \textcite{mayhew1995allocation}, \textcite{fleming1996trading}, among others, argue that widened financial leverage and narrowed transaction costs may encourage informed traders to trade in the option market instead of the equity market,
Third, unlike stock market, there are no short selling constraints in option market. Therefore, the characteristics of an option contract are informational and worthy to investigate. However, To what extent has it been supported by empirical works? 

It has been a fierce debate that the trading information from derivative markets lead the underlying markets. Prior studies hold different arguments toward this topic. Relevant papers, like \textcite{manaster1982option}, \textcite{anthony1988interrelation}, \textcite{chakravarty2004informed}, 
\textcite{cremers2010deviations}, 
\textcite{xing2010does} have found that the information from option market take the lead of the information from stock market. 
When informed traders recieved private information, they prefer to trade in option market since there are several advatanges we mentioned in the first paragraph. Furthermore, voluminious literatures address that the informed trading in option market elucidate the process of price discovery, and the information contained in option price and volume would eventually get incoperated into the underlying prices.
While, other studies like 
\textcite{chan1993option}, 
\textcite{stephan1990intraday} find no evidence that option prices can lead stock prices. 

In this paper, we provide a comprehensive analysis on the examination of the interrelation between option and index markets. In other words, this paper contributes to the literature in several ways. First, rather than investigating on stock market, we focus mainly on the index, S\&P 500. Since we would like to research on which interval in a single trading day carries the trading information the most rather than study imbalance orders behind an option volume, index are harder for traders to acquire private information than others. Most prior literature discuss informational linkage between option and stock markets, few of them had mentioned about index market.   


Based on the call and put implied volatilty spread(CPIV) propsed by \textcite{cremers2010deviations}, we refine this approach and derive an intraday version which tells the predictability power toward future index returns within a single trading day. Within a single day, we would like to see whether the predictability power is stronger in open period, middle period or close period. Therefore, we partition a single day to 14 intervals, and each interval we only include 5 minutes long in case the Put-Call parity would be unbalanced due to major difference between the underlying prices in call and put options. Apart from the above reason, there are several causes that are responsible for the deviation of put-call parity, short sell constraints, the early exercise value of American options, transaction costs, taxes, to name a few. 

The remainder of paper is organized as follows. In Section 2 we describe our research hypothesis, In Section 3, we describe our methedology and data. Section 4 presents the main empirical results on quote data on predicting index returns. Section 5 provides the results are robust to trade data given the identical sample period. Section 6 concludes. 