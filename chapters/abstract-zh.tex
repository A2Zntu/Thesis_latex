\section*{\centerline{\textbf{\huge 摘要}}}
\addcontentsline{toc}{section}{摘要}

\noindent 


過去許多文獻在探討選擇權隱含波動度偏離與未來標的報酬的關係時採用的是日資料,本研究採用了日內資料以觀察在同一個交易日中,是否有其他的時間區段對於大盤報酬有正向關係。研究發現,在開盤與盤中時的選擇權隱含波動度偏離對於當日的報酬有解釋能力,且其解釋能力在 (i) 極端消費者情緒的區間 (ii) 選擇權流動性高的區間 (iii) 經濟衰退 時更為顯著。另外, 本研究也利用不同區段的選擇權隱含波動度偏離對日內的大盤報酬進行預測,發現開盤區段的波動度偏離相對其他時間含有更多交易資訊,且其預測能力隨著交易時間而遞減。
\vspace{1cm}


\noindent{\bf 關鍵字:}S\&P 500、選擇權報酬、隱含波動度差、日內資料、投資者情緒、選擇權流動性


