
In recent year, the information of the derivatives market plays an increasingly crucial role in financial markets. There are multiple reasons why the option market is vital for investors. First, informed traders may choose to trade in derivative markets since they can camouflage with noise traders by trading different options contracts on one specific security \parencite{easley1998option}. Second, authors such as \textcite{black1973pricing}, \textcite{mayhew1995allocation}, \textcite{fleming1996trading}, among others, argue that widened financial leverage and narrowed transaction costs may encourage informed traders to trade in the option market instead of the equity market. Third, unlike the stock market, there are no short-selling constraints in option market. Therefore, the characteristics of an option contract are informational and worthy to investigate. However, to what extent has it been supported by empirical works? 

It has been a fierce debate on whether the trading information from derivative markets leads the underlying markets. Prior studies hold different arguments toward this topic. Relevant papers, like \textcite{manaster1982option}, \textcite{anthony1988interrelation}, \textcite{chakravarty2004informed}, \textcite{cremers2010deviations}, \textcite{xing2010does} have found that the information from option market takes the lead of the information from stock market. When informed traders receive private information, they prefer to trade in options market since there are several advantages we mentioned in the first paragraph. Furthermore, plentiful literature address that the informed trading in options market elucidates the process of price discovery, and the information contained in option price and volume would eventually get incorporated into underlying prices. However, other studies, like \textcite{chan1993option}, \textcite{stephan1990intraday} find no evidence that option prices can lead to stock prices. 

Nevertheless, after the sequential model built in \textcite{easley1998option}, myriad literature tried to capture the information from options market. Studies like \textcite{doran2007there}, \textcite{doran2010implications} and \textcite{atilgan2015implied} examined the impact on future asset returns of information contained in the implied volatility skewness. Others like \textcite{chan2002informational} and \textcite{pan2006information} found an ordered imbalance in options volume facilitates the predictability in future asset returns. Apart from that, \textcite{bali2009volatility} and \textcite{cremers2010deviations} employ the call-put implied volatility spread (CPIV) to construct portfolios. Moreover, they prove that the deviations of put-call parity contain information about future stock returns. By longing the highest CPIV portfolio and shorting the lowest CPIV portfolio can generate a positive and significant alpha.   

In this paper, we provide a comprehensive analysis of the examination of the interrelation between option and index markets. First, rather than investigating on the stock market, we focus mainly on the index -- S\&P 500. Since most prior literature discusses the informational linkage between option and stock markets, few of them had mentioned about the index market. One literature, \textcite{atilgan2015implied},  has investigated on S\&P 500 option and index market; however, it employs the way of implied volatility skewness rather than implied volatility spreads. We focus on the latter and discover different results from them. In addition, another reason we choose to explore S\&P 500 rather than individual stocks is that index is not allowed to short comparing with stocks, so we believe there are more incentives for investors to trade in options market first from the aspect of liquidity.

Different from the previous studies primarily deliberating about the option information on daily frequency, we would like to investigate the experiments on intraday frequency. Furthermore, prior studies data are mainly constructed by "highest closing bid price" and "lowest closing ask price", which only capture the option information of 5-minutes interval before options market close. Therefore, we would like to test whether the predictability power is stronger in the open period, middle period or close period.
         
Based on the CPIV proposed by \textcite{cremers2010deviations}, we refine this approach and derive an intraday version. We partition a single trading day to 14 intervals, and each interval we only include 5-minutes long data in case the put-call parity would be unbalanced due to the major difference between the underlying prices in call and put options\footnote{Apart from the above reason, there are several causes that are responsible for the deviation of put-call parity, short sell constraints, the early exercise value of American options, transaction costs, taxes, to name a few}. In each interval, we examine all the possible option pairs and pick up the valid ones to compose a representative CPIV.

Our main results are simply summarized. To start with, the results display an intermarket relation between quote-data CPIV and contemporaneous index returns. However, there is no predictive power toward one-day ahead index returns\footnote{We also practice a robustness check on trade-data CPIV. Surprisingly, the $CPIV_{10:30}$ has strong predictability to forecast a one-day-ahead index return}. We conclude that these intraday CPIV are enormously informational on index returns during the identical day, but once the market close and open again in next day, the price of S\&P 500 would soon be adjusted during the first couple minutes. Therefore, it is even harder to generate an information gap from insider to noise trader.  

Next, we focus on intraday index returns, and the empirical results express the open interval implied volatility spreads -- $CPIV_{08:30}$, has intense predictability toward the intraday half-hours cumulated returns. The adjusted $R^{2}$ range from 50.5\% to 23.6\% as hours decay during the rest of the day. All other intervals CPIV have no such powerful predictability as $CPIV_{08:30}$, the result again proves that the most informational interval is the first interval no matter on the daily frequency and intraday frequency.     

Finally, we also conduct tests to yield further evidence for our information-based hypothesis. Following \textcite{atilgan2015implied}, our finding presents that the periods with high sentiment index are more sensitive to information distributing between markets. Hence, the relation between CPIV and returns is significantly stronger during the particular periods as previous studies suggested. In \textcite{pan2006information}, \textcite{chang2018implied}, they also examine the tests on different stock liquidity and option liquidity periods. They all agree that during the periods with low stock liquidity or high option liquidity, the predictability on future asset returns increases. We segregate the periods by option liquidity, and the empirical results demonstrate the above ideas, where the coefficients of interval CPIV are larger and positive in high option liquidity periods. Overall, these results convey that information explanation is supported by empirical analysis. 

The remainder of our research is organized as follows. In Section 2 we describe our research hypothesis, In Section 3, we describe our methodology and data. Section 4 presents the main empirical results on predicting index returns. Section 5 provides additional evidence for the information explanation. Section 6 concludes. 